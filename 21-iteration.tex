\documentclass[]{article}
\usepackage{lmodern}
\usepackage{amssymb,amsmath}
\usepackage{ifxetex,ifluatex}
\usepackage{fixltx2e} % provides \textsubscript
\ifnum 0\ifxetex 1\fi\ifluatex 1\fi=0 % if pdftex
  \usepackage[T1]{fontenc}
  \usepackage[utf8]{inputenc}
\else % if luatex or xelatex
  \ifxetex
    \usepackage{mathspec}
  \else
    \usepackage{fontspec}
  \fi
  \defaultfontfeatures{Ligatures=TeX,Scale=MatchLowercase}
\fi
% use upquote if available, for straight quotes in verbatim environments
\IfFileExists{upquote.sty}{\usepackage{upquote}}{}
% use microtype if available
\IfFileExists{microtype.sty}{%
\usepackage{microtype}
\UseMicrotypeSet[protrusion]{basicmath} % disable protrusion for tt fonts
}{}
\usepackage[margin=1in]{geometry}
\usepackage{hyperref}
\hypersetup{unicode=true,
            pdfborder={0 0 0},
            breaklinks=true}
\urlstyle{same}  % don't use monospace font for urls
\usepackage{color}
\usepackage{fancyvrb}
\newcommand{\VerbBar}{|}
\newcommand{\VERB}{\Verb[commandchars=\\\{\}]}
\DefineVerbatimEnvironment{Highlighting}{Verbatim}{commandchars=\\\{\}}
% Add ',fontsize=\small' for more characters per line
\usepackage{framed}
\definecolor{shadecolor}{RGB}{248,248,248}
\newenvironment{Shaded}{\begin{snugshade}}{\end{snugshade}}
\newcommand{\KeywordTok}[1]{\textcolor[rgb]{0.13,0.29,0.53}{\textbf{#1}}}
\newcommand{\DataTypeTok}[1]{\textcolor[rgb]{0.13,0.29,0.53}{#1}}
\newcommand{\DecValTok}[1]{\textcolor[rgb]{0.00,0.00,0.81}{#1}}
\newcommand{\BaseNTok}[1]{\textcolor[rgb]{0.00,0.00,0.81}{#1}}
\newcommand{\FloatTok}[1]{\textcolor[rgb]{0.00,0.00,0.81}{#1}}
\newcommand{\ConstantTok}[1]{\textcolor[rgb]{0.00,0.00,0.00}{#1}}
\newcommand{\CharTok}[1]{\textcolor[rgb]{0.31,0.60,0.02}{#1}}
\newcommand{\SpecialCharTok}[1]{\textcolor[rgb]{0.00,0.00,0.00}{#1}}
\newcommand{\StringTok}[1]{\textcolor[rgb]{0.31,0.60,0.02}{#1}}
\newcommand{\VerbatimStringTok}[1]{\textcolor[rgb]{0.31,0.60,0.02}{#1}}
\newcommand{\SpecialStringTok}[1]{\textcolor[rgb]{0.31,0.60,0.02}{#1}}
\newcommand{\ImportTok}[1]{#1}
\newcommand{\CommentTok}[1]{\textcolor[rgb]{0.56,0.35,0.01}{\textit{#1}}}
\newcommand{\DocumentationTok}[1]{\textcolor[rgb]{0.56,0.35,0.01}{\textbf{\textit{#1}}}}
\newcommand{\AnnotationTok}[1]{\textcolor[rgb]{0.56,0.35,0.01}{\textbf{\textit{#1}}}}
\newcommand{\CommentVarTok}[1]{\textcolor[rgb]{0.56,0.35,0.01}{\textbf{\textit{#1}}}}
\newcommand{\OtherTok}[1]{\textcolor[rgb]{0.56,0.35,0.01}{#1}}
\newcommand{\FunctionTok}[1]{\textcolor[rgb]{0.00,0.00,0.00}{#1}}
\newcommand{\VariableTok}[1]{\textcolor[rgb]{0.00,0.00,0.00}{#1}}
\newcommand{\ControlFlowTok}[1]{\textcolor[rgb]{0.13,0.29,0.53}{\textbf{#1}}}
\newcommand{\OperatorTok}[1]{\textcolor[rgb]{0.81,0.36,0.00}{\textbf{#1}}}
\newcommand{\BuiltInTok}[1]{#1}
\newcommand{\ExtensionTok}[1]{#1}
\newcommand{\PreprocessorTok}[1]{\textcolor[rgb]{0.56,0.35,0.01}{\textit{#1}}}
\newcommand{\AttributeTok}[1]{\textcolor[rgb]{0.77,0.63,0.00}{#1}}
\newcommand{\RegionMarkerTok}[1]{#1}
\newcommand{\InformationTok}[1]{\textcolor[rgb]{0.56,0.35,0.01}{\textbf{\textit{#1}}}}
\newcommand{\WarningTok}[1]{\textcolor[rgb]{0.56,0.35,0.01}{\textbf{\textit{#1}}}}
\newcommand{\AlertTok}[1]{\textcolor[rgb]{0.94,0.16,0.16}{#1}}
\newcommand{\ErrorTok}[1]{\textcolor[rgb]{0.64,0.00,0.00}{\textbf{#1}}}
\newcommand{\NormalTok}[1]{#1}
\usepackage{graphicx,grffile}
\makeatletter
\def\maxwidth{\ifdim\Gin@nat@width>\linewidth\linewidth\else\Gin@nat@width\fi}
\def\maxheight{\ifdim\Gin@nat@height>\textheight\textheight\else\Gin@nat@height\fi}
\makeatother
% Scale images if necessary, so that they will not overflow the page
% margins by default, and it is still possible to overwrite the defaults
% using explicit options in \includegraphics[width, height, ...]{}
\setkeys{Gin}{width=\maxwidth,height=\maxheight,keepaspectratio}
\IfFileExists{parskip.sty}{%
\usepackage{parskip}
}{% else
\setlength{\parindent}{0pt}
\setlength{\parskip}{6pt plus 2pt minus 1pt}
}
\setlength{\emergencystretch}{3em}  % prevent overfull lines
\providecommand{\tightlist}{%
  \setlength{\itemsep}{0pt}\setlength{\parskip}{0pt}}
\setcounter{secnumdepth}{0}
% Redefines (sub)paragraphs to behave more like sections
\ifx\paragraph\undefined\else
\let\oldparagraph\paragraph
\renewcommand{\paragraph}[1]{\oldparagraph{#1}\mbox{}}
\fi
\ifx\subparagraph\undefined\else
\let\oldsubparagraph\subparagraph
\renewcommand{\subparagraph}[1]{\oldsubparagraph{#1}\mbox{}}
\fi

%%% Use protect on footnotes to avoid problems with footnotes in titles
\let\rmarkdownfootnote\footnote%
\def\footnote{\protect\rmarkdownfootnote}

%%% Change title format to be more compact
\usepackage{titling}

% Create subtitle command for use in maketitle
\newcommand{\subtitle}[1]{
  \posttitle{
    \begin{center}\large#1\end{center}
    }
}

\setlength{\droptitle}{-2em}
  \title{}
  \pretitle{\vspace{\droptitle}}
  \posttitle{}
  \author{}
  \preauthor{}\postauthor{}
  \date{}
  \predate{}\postdate{}


\begin{document}

\section*{Chapter 21 - Iteration}\label{chapter-21---iteration}
\addcontentsline{toc}{section}{Chapter 21 - Iteration}

\begin{Shaded}
\begin{Highlighting}[]
\KeywordTok{library}\NormalTok{(tidyverse)}
\KeywordTok{library}\NormalTok{(stringr)}
\end{Highlighting}
\end{Shaded}

\subsection*{21.2 For loops}\label{for-loops}
\addcontentsline{toc}{subsection}{21.2 For loops}

\subsubsection*{Problem 1}\label{problem-1}
\addcontentsline{toc}{subsubsection}{Problem 1}

Write a for loop to compute the mean of every column in mtcars.

\begin{Shaded}
\begin{Highlighting}[]
\NormalTok{x <-}\StringTok{ }\KeywordTok{vector}\NormalTok{(}\StringTok{"double"}\NormalTok{, }\KeywordTok{ncol}\NormalTok{(mtcars))}
\ControlFlowTok{for}\NormalTok{ (i }\ControlFlowTok{in} \KeywordTok{seq_along}\NormalTok{(mtcars)) \{}
\NormalTok{  x[[i]] <-}\StringTok{ }\KeywordTok{mean}\NormalTok{(mtcars[[i]])}
\NormalTok{\}}
\end{Highlighting}
\end{Shaded}

Write a for loop to determine the type of each column in
nycflights13::flights. (NOTE: Unsure how to deal with class ``unknown''
in column 19)

\begin{Shaded}
\begin{Highlighting}[]
\NormalTok{x <-}\StringTok{ }\KeywordTok{vector}\NormalTok{(}\StringTok{"character"}\NormalTok{, }\KeywordTok{ncol}\NormalTok{(nycflights13}\OperatorTok{::}\NormalTok{flights))}
\ControlFlowTok{for}\NormalTok{ (i }\ControlFlowTok{in} \KeywordTok{seq_along}\NormalTok{(nycflights13}\OperatorTok{::}\NormalTok{flights)) \{}
\NormalTok{  x[[i]] <-}\StringTok{ }\KeywordTok{str_c}\NormalTok{(}\KeywordTok{class}\NormalTok{(nycflights13}\OperatorTok{::}\NormalTok{flights[[i]]), }\DataTypeTok{collapse =} \StringTok{", "}\NormalTok{)}
\NormalTok{\}}
\end{Highlighting}
\end{Shaded}

Write a for loop to compute the number of unique values in each column
of iris.

\begin{Shaded}
\begin{Highlighting}[]
\NormalTok{x <-}\StringTok{ }\KeywordTok{vector}\NormalTok{(}\StringTok{"integer"}\NormalTok{, }\KeywordTok{ncol}\NormalTok{(iris))}
\ControlFlowTok{for}\NormalTok{ (i }\ControlFlowTok{in} \KeywordTok{seq_along}\NormalTok{(iris)) \{}
\NormalTok{  x[[i]] <-}\StringTok{ }\KeywordTok{length}\NormalTok{(}\KeywordTok{unique}\NormalTok{(iris[[i]]))}
\NormalTok{\}}
\end{Highlighting}
\end{Shaded}

Write a for loop to generate 10 random normals for each of mean = -10,
0, 10, 100.

\begin{Shaded}
\begin{Highlighting}[]
\NormalTok{means <-}\StringTok{ }\KeywordTok{c}\NormalTok{(}\KeywordTok{rep}\NormalTok{(}\OperatorTok{-}\DecValTok{10}\NormalTok{, }\DecValTok{10}\NormalTok{), }\KeywordTok{rep}\NormalTok{(}\DecValTok{0}\NormalTok{, }\DecValTok{10}\NormalTok{), }\KeywordTok{rep}\NormalTok{(}\DecValTok{10}\NormalTok{, }\DecValTok{10}\NormalTok{), }\KeywordTok{rep}\NormalTok{(}\DecValTok{100}\NormalTok{, }\DecValTok{10}\NormalTok{))}
\NormalTok{x <-}\StringTok{ }\KeywordTok{vector}\NormalTok{(}\StringTok{"double"}\NormalTok{, }\DecValTok{40}\NormalTok{)}
\ControlFlowTok{for}\NormalTok{ (i }\ControlFlowTok{in} \KeywordTok{seq_along}\NormalTok{(means)) \{}
\NormalTok{  x[[i]] <-}\StringTok{ }\KeywordTok{rnorm}\NormalTok{(}\DecValTok{1}\NormalTok{, means[[i]])}
\NormalTok{\}}
\end{Highlighting}
\end{Shaded}

\subsubsection*{Problem 2}\label{problem-2}
\addcontentsline{toc}{subsubsection}{Problem 2}

Eliminate the for loop in each of the following examples by taking
advantage of an existing function that works with vectors:

\begin{Shaded}
\begin{Highlighting}[]
\NormalTok{out <-}\StringTok{ ""}
\ControlFlowTok{for}\NormalTok{ (x }\ControlFlowTok{in}\NormalTok{ letters) \{}
\NormalTok{  out <-}\StringTok{ }\NormalTok{stringr}\OperatorTok{::}\KeywordTok{str_c}\NormalTok{(out, x)}
\NormalTok{\}}

\KeywordTok{str_c}\NormalTok{(letters[}\DecValTok{1}\OperatorTok{:}\DecValTok{26}\NormalTok{], }\DataTypeTok{collapse =} \StringTok{""}\NormalTok{)}
\end{Highlighting}
\end{Shaded}

\begin{verbatim}
## [1] "abcdefghijklmnopqrstuvwxyz"
\end{verbatim}

\begin{Shaded}
\begin{Highlighting}[]
\NormalTok{x <-}\StringTok{ }\KeywordTok{sample}\NormalTok{(}\DecValTok{100}\NormalTok{)}
\NormalTok{sd <-}\StringTok{ }\DecValTok{0}
\ControlFlowTok{for}\NormalTok{ (i }\ControlFlowTok{in} \KeywordTok{seq_along}\NormalTok{(x)) \{}
\NormalTok{  sd <-}\StringTok{ }\NormalTok{sd }\OperatorTok{+}\StringTok{ }\NormalTok{(x[i] }\OperatorTok{-}\StringTok{ }\KeywordTok{mean}\NormalTok{(x)) }\OperatorTok{^}\StringTok{ }\DecValTok{2}
\NormalTok{\}}
\NormalTok{sd <-}\StringTok{ }\KeywordTok{sqrt}\NormalTok{(sd }\OperatorTok{/}\StringTok{ }\NormalTok{(}\KeywordTok{length}\NormalTok{(x) }\OperatorTok{-}\StringTok{ }\DecValTok{1}\NormalTok{))}

\KeywordTok{sd}\NormalTok{(x)}
\end{Highlighting}
\end{Shaded}

\begin{verbatim}
## [1] 29.01149
\end{verbatim}

\begin{Shaded}
\begin{Highlighting}[]
\NormalTok{x <-}\StringTok{ }\KeywordTok{runif}\NormalTok{(}\DecValTok{100}\NormalTok{)}
\NormalTok{out <-}\StringTok{ }\KeywordTok{vector}\NormalTok{(}\StringTok{"numeric"}\NormalTok{, }\KeywordTok{length}\NormalTok{(x))}
\NormalTok{out[}\DecValTok{1}\NormalTok{] <-}\StringTok{ }\NormalTok{x[}\DecValTok{1}\NormalTok{]}
\ControlFlowTok{for}\NormalTok{ (i }\ControlFlowTok{in} \DecValTok{2}\OperatorTok{:}\KeywordTok{length}\NormalTok{(x)) \{}
\NormalTok{  out[i] <-}\StringTok{ }\NormalTok{out[i }\OperatorTok{-}\StringTok{ }\DecValTok{1}\NormalTok{] }\OperatorTok{+}\StringTok{ }\NormalTok{x[i]}
\NormalTok{\}}

\KeywordTok{cumsum}\NormalTok{(x)}
\end{Highlighting}
\end{Shaded}

\begin{verbatim}
##   [1]  0.7126368  1.1428565  1.3350483  2.0863673  2.4616929  2.5740213
##   [7]  3.4734077  4.1502276  4.3822943  5.2064431  5.7511704  6.4447496
##  [13]  6.6698051  7.1222354  7.3456017  7.4092191  7.8859488  8.4777228
##  [19]  8.6182507  8.7582410  9.2448489  9.7156475 10.4643351 10.6283577
##  [25] 11.0299327 11.7980679 12.3791622 12.9448362 13.5414886 13.9951206
##  [31] 14.0025174 14.5530792 14.7783474 15.2152726 16.0999847 16.5945454
##  [37] 16.6820458 17.3556377 17.8459867 18.6177203 18.6954417 19.1405699
##  [43] 19.2081645 19.4867348 20.3267390 20.6012557 20.6509275 21.6137493
##  [49] 22.3308028 23.2101229 23.6290082 23.8356498 24.2824199 24.5575691
##  [55] 24.9885431 25.1508428 25.9070173 26.6794756 27.1847945 27.6380926
##  [61] 28.3821794 29.3449061 30.2459320 30.9094035 31.7809494 31.9110324
##  [67] 32.6152725 32.8892323 33.2425951 33.5249680 33.5734816 34.0868509
##  [73] 34.2556986 34.5657226 34.9095146 35.5799226 35.8703644 36.6124394
##  [79] 36.8233828 37.5271252 38.1227696 39.0822185 39.6787612 39.7719824
##  [85] 40.3390825 40.7900905 41.7884477 41.8183248 42.3681158 43.1963537
##  [91] 43.3175848 43.7613808 43.9341762 44.0915534 44.9966318 45.5807904
##  [97] 46.1307274 46.6302637 46.9275776 47.2325942
\end{verbatim}

\subsubsection*{Problem 3}\label{problem-3}
\addcontentsline{toc}{subsubsection}{Problem 3}

Write a for loop that prints() the lyrics to the children's song ``Alice
the camel''.

\begin{Shaded}
\begin{Highlighting}[]
\NormalTok{humps <-}\StringTok{ }\KeywordTok{c}\NormalTok{(}\StringTok{"five"}\NormalTok{, }\StringTok{"four"}\NormalTok{, }\StringTok{"three"}\NormalTok{, }\StringTok{"two"}\NormalTok{, }\StringTok{"one"}\NormalTok{, }\StringTok{"no"}\NormalTok{)}

\NormalTok{y <-}\StringTok{ }\KeywordTok{vector}\NormalTok{(}\StringTok{"character"}\NormalTok{, }\DecValTok{6}\NormalTok{)}
\ControlFlowTok{for}\NormalTok{ (x }\ControlFlowTok{in} \KeywordTok{seq_along}\NormalTok{(humps)) \{}
\NormalTok{  x <-}\StringTok{ }\KeywordTok{str_replace_all}\NormalTok{(}\KeywordTok{str_c}\NormalTok{(}\KeywordTok{c}\NormalTok{(}\KeywordTok{rep}\NormalTok{(}\StringTok{"Alice the camel has x humps}\CharTok{\textbackslash{}n}\StringTok{"}\NormalTok{, }\DecValTok{3}\NormalTok{), }\StringTok{"So go, Alice, go.}\CharTok{\textbackslash{}n}\StringTok{"}\NormalTok{), }\DataTypeTok{collapse =} \StringTok{""}\NormalTok{),}
                             \StringTok{"x"}\NormalTok{, humps)}
\NormalTok{\}}
\KeywordTok{print}\NormalTok{(}\KeywordTok{str_c}\NormalTok{(}\KeywordTok{c}\NormalTok{(x, }\StringTok{"Now Alice has no humps!"}\NormalTok{), }\DataTypeTok{collapse =} \StringTok{""}\NormalTok{))}
\end{Highlighting}
\end{Shaded}

\begin{verbatim}
## [1] "Alice the camel has five humps\nAlice the camel has five humps\nAlice the camel has five humps\nSo go, Alice, go.\nAlice the camel has four humps\nAlice the camel has four humps\nAlice the camel has four humps\nSo go, Alice, go.\nAlice the camel has three humps\nAlice the camel has three humps\nAlice the camel has three humps\nSo go, Alice, go.\nAlice the camel has two humps\nAlice the camel has two humps\nAlice the camel has two humps\nSo go, Alice, go.\nAlice the camel has one humps\nAlice the camel has one humps\nAlice the camel has one humps\nSo go, Alice, go.\nAlice the camel has no humps\nAlice the camel has no humps\nAlice the camel has no humps\nSo go, Alice, go.\nNow Alice has no humps!"
\end{verbatim}

Convert the nursery rhyme ``ten in the bed'' to a function. Generalise
it to any number of people in any sleeping structure.

\begin{Shaded}
\begin{Highlighting}[]
\NormalTok{numbs <-}\StringTok{ }\KeywordTok{c}\NormalTok{(}\StringTok{"five"}\NormalTok{, }\StringTok{"four"}\NormalTok{, }\StringTok{"three"}\NormalTok{, }\StringTok{"two"}\NormalTok{)}
\NormalTok{y <-}\StringTok{ }\KeywordTok{vector}\NormalTok{(}\StringTok{"character"}\NormalTok{, }\DecValTok{5}\NormalTok{)}
\ControlFlowTok{for}\NormalTok{ (x }\ControlFlowTok{in} \KeywordTok{seq_along}\NormalTok{(numbs)) \{}
\NormalTok{  y <-}\StringTok{ }\KeywordTok{str_c}\NormalTok{(}\KeywordTok{str_c}\NormalTok{(}\StringTok{"There were "}\NormalTok{,numbs,}\StringTok{" in the bed}\CharTok{\textbackslash{}n}\StringTok{And the little one said,}\CharTok{\textbackslash{}n}\StringTok{'Roll over! Roll over!'}\CharTok{\textbackslash{}n}\StringTok{So they all rolled over and}\CharTok{\textbackslash{}n}\StringTok{one fell out."}\NormalTok{), }\StringTok{"}\CharTok{\textbackslash{}n}\StringTok{There was one in the bed}\CharTok{\textbackslash{}n}\StringTok{And the little one said,}\CharTok{\textbackslash{}n}\StringTok{Alone at last!"}\NormalTok{)}
\NormalTok{\}}
\end{Highlighting}
\end{Shaded}


\end{document}
